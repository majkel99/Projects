\documentclass[12pt,a4paper]{article}
\usepackage[polish]{babel}
\usepackage[T1]{fontenc}
\usepackage[utf8x]{inputenc}
\usepackage{hyperref}
\usepackage{url}
\usepackage[]{algorithm2e}
\usepackage{listings}
\usepackage{graphicx}
\usepackage{algpseudocode}

\usepackage{color}
\usepackage{listings}

\lstloadlanguages{% Check Dokumentation for further languages ...
	C,
	C++,
	csh,
	Java
}

\definecolor{red}{rgb}{0.6,0,0} % for strings
\definecolor{blue}{rgb}{0,0,0.6}
\definecolor{green}{rgb}{0,0.8,0}
\definecolor{cyan}{rgb}{0.0,0.6,0.6}

\lstset{
	language=csh,
	basicstyle=\footnotesize\ttfamily,
	numbers=left,
	numberstyle=\tiny,
	numbersep=5pt,
	tabsize=2,
	extendedchars=true,
	breaklines=true,
	frame=b,
	stringstyle=\color{blue}\ttfamily,
	showspaces=false,
	showtabs=false,
	xleftmargin=17pt,
	framexleftmargin=17pt,
	framexrightmargin=5pt,
	framexbottommargin=4pt,
	commentstyle=\color{green},
	morecomment=[l]{//}, %use comment-line-style!
	morecomment=[s]{/*}{*/}, %for multiline comments
	showstringspaces=false,
	morekeywords={ abstract, event, new, struct,
		as, explicit, null, switch,
		base, extern, object, this,
		bool, false, operator, throw,
		break, finally, out, true,
		byte, fixed, override, try,
		case, float, params, typeof,
		catch, for, private, uint,
		char, foreach, protected, ulong,
		checked, goto, public, unchecked,
		class, if, readonly, unsafe,
		const, implicit, ref, ushort,
		continue, in, return, using,
		decimal, int, sbyte, virtual,
		default, interface, sealed, volatile,
		delegate, internal, short, void,
		do, is, sizeof, while,
		double, lock, stackalloc,
		else, long, static,
		enum, namespace, string},
	keywordstyle=\color{cyan},
	identifierstyle=\color{red},
}
\usepackage{caption}
\DeclareCaptionFont{white}{\color{white}}
\DeclareCaptionFormat{listing}{\colorbox{blue}{\parbox{\textwidth}{\hspace{15pt}#1#2#3}}}
\captionsetup[lstlisting]{format=listing,labelfont=white,textfont=white, singlelinecheck=false, margin=0pt, font={bf,footnotesize}}


\addtolength{\hoffset}{-1.5cm}
\addtolength{\marginparwidth}{-1.5cm}
\addtolength{\textwidth}{3cm}
\addtolength{\voffset}{-1cm}
\addtolength{\textheight}{2.5cm}
\setlength{\topmargin}{0cm}
\setlength{\headheight}{0cm}

\begin{document}
	
	\title{Programowanie III\\{dokumentacja projektu Statki}}
	\author{Michał Pluszczewski grupa 4G\\Politechnika Śląska \\ Wydział: Matematyki Stosowanej\\ Kierunek: Informatyka}

	\date{\today}

	\maketitle
	\newpage
	\section*{Część I}
	\subsection*{Opis programu Statki}
	W grze w statki należy przygotować planszę wielkości 10 na 10 kratek, na której
rozmieszczone mają być: cztery jednomasztowce (jedna kratka), trzy dwumasztowce
(dwie kratki), trzy trójmasztowce (trzy kratki) i jeden czteromasztowiec (cztery kratki).
Zasady rozmieszczania statków na planszy są następujące:\\
a) pola dwumasztowców, trójmasztowców i czteromasztowca muszą być tak
umieszczone na planszy, aby z każdego pola danego statku dało się przejść do
każdego innego pola tego statku przechodząc wyłącznie przez ich wspólne
boki;\\
b) dwa różne statki nie mogą się stykać ze sobą ani bokami, ani wierzchołkami.
Napisz program, który zwracał będzie losowe i poprawne rozmieszczenie statków
na planszy. Plansza ma być wyświetlana ma monitorze, a statki zaznaczone mają być
na niej znakami ,,x’’.
	\section*{Część II}
	\subsection*{Opis działania}
	\begin{center}
	Program nie przyjmuje żadnych danych wejściowych.\\
    \includegraphics{Screenshot_4.png}\\
    Rysunek 1: Program po uruchomieniu
    \end{center}
    Program wypisuje w losowym punkcie na planszy w konsoli cztery jednomasztowce (jedna kratka), trzy dwumasztowce (dwie kratki), trzy
trójmasztowce (trzy kratki) i jeden czteromasztowiec (cztery kratki), które nie mogą się ze sobą stykać ani bokami, ani wierzchołkami. 
    \newpage
	\subsection*{Algorytm}
	Program jest podzielony na kilka funkcji, umieszczonych w klasie Statki.\\
	\\
	W funkcji wypelnienie planszy wypełniamy naszą planszę pustymi znakami -.\\
	\\
	W funkcji wypisanie planszy wypisujemy planszę na ekran.\\
	\\
	W funkcji getRandomNumber losujemy losową liczbę z danego przedziału.\\
	\\
	W funkcji LosowaniePunktu losujemy dwie liczby od 0 do 9 oraz kierunek w którym wypiszemy dany
	masztowiec (liczby od 0 do 3). Następnie wywołujemy funkcję sprawdz punkt i jeśli funkcja ta zwróci wartość true przechodzimy do wypisywania naszego masztowca w odpowiednim miejscu na planszy.\\
	\\
	W funkcji sprawdz punkty sprawdzamy czy możemy zapisać masztowiec na planszy(jego otoczenie jak i miejsce w którym chcemy go zapisać). Jeśli choć jedno sprawdzane pole jest różne od znaku - funkcja zwraca wartość false. Natomiast jeśli wszystkie warunki są spełnione zwaraca wartość true.\\
	\\
	W funkcji main tworzymy plansze znaków 10 na 10, następnie wypełniamy planszę poprzez wysowałnie odpowiedniej funkcji, po czym odpowiednio dla rozmiaru i ilości masztowców losujemy i sprawdzamy punkty wywołując funkcję LosowaniePunktu. Wypisujemy planszę na ekran wywołując funkcję wypisanie planszy.\\
	\\
	Schemat blokowy programu:
	\begin{center}
	    \includegraphics{main.png}
	    Rysunek 2: Schemat blokowy funkcji main
	\end{center}
	\begin{center}
	    \includegraphics{wypelnienie_planszy.png}
	    \\
	    Rysunek 3: Schemat blokowy funkcji wypelnienie planszy
	\end{center}
	\begin{center}
	    \includegraphics{wypisanie_planszy.png}
	    \\
	    Rysunek 4: Schemat blokowy funkcji wypisanie planszy
	\end{center}
	\begin{center}
	    \includegraphics{LosowaniePunktu.png}
	    \includegraphics{LosowaniePunktu2.png}
	    \\
	    Rysunek 4: Schemat blokowy funkcji LosowaniePunktu
	\end{center}
	\begin{center}
	    \includegraphics{sprawdz_punkt.png}
	    \includegraphics{sprawdz_punkt2.png}
	    \\
	    Rysunek 4: Schemat blokowy funkcji sprawdz punkt
	    \\
	    Sytuacja wygląda analogicznie dla innych kierunków
	\end{center}
	
    Pseudokod:\\
	\begin{algorithm}[H]
		\KwData{Dane wejściowe: plansza znaków 10 na 10, rozmiar masztowca}
		\While{$true$}{
		    \While{$true$}{
		        $x=getRandomNumber(0,9)$;\\
		        $y=getRandomNumber(0,9)$;\\
		        $kierunek=getRandomNumber(0,3)$\\
		        \If{(sprawdzpunkt(plansza,x,y,kierunek,rozmiarm)==$true$)}{
		            $break$;
		        }
		    }
			\If{(kierunek==0 and (x >= 0 and x <= 9) and (y >= 0 and y$+$rozmiarm < 9))}{
			    \For{int i = 0; i < rozmiarm; i++}{
			        $plansza[x][y + i]$ = 'x';
			    }
			    break;
			}
			\If{(kierunek==1 and (x >= 0 and x <= 9) and (y$-$rozmiarm > 0 and y <= 9))}{
			    \For{int i = 0; i < rozmiarm; i++}{
			        $plansza[x][y - i]$ = 'x';
			    }
			    break;
			}
			\If{(kierunek==2 and (x$-$rozmiarm > 0 and x <= 9) and (y >= 0 and y <= 9))}{
			    \For{int i = 0; i < rozmiarm; i++}{
			        $plansza[x][y + i]$ = 'x';
			    }
			    break;
			}
			\If{(kierunek==3 and (x >= 0 and x$+$rozmiarm < 9) and (y >= 0 and y <= 9))}{
			    \For{int i = 0; i < rozmiarm; i++}{
			        $plansza[x][y + i]$ = 'x';
			    }
			    break;
			}
		}
		\caption{Algorytm losowania punktu.}
	\end{algorithm}
	\newpage
	\subsection*{Implementacja}
	\newline\newline
	Implementacja pseudokodu napisanego w subsekcji Algorytm
	\begin{lstlisting}
public static void LosowaniePunktu(char[][] plansza, int rozmiar_m){
    int x, y, kierunek;
    while(true){
        while(true){
            x=getRandomNumber(0,9);
            y=getRandomNumber(0,9);
            kierunek=getRandomNumber(0,3);
            if (sprawdz_punkt(plansza,x,y,kierunek,rozmiar_m))
                break;          
        }
      //prawo
        if (kierunek == 0 && (x >= 0 && x <= 9) && (y >= 0 && y+rozmiar_m < 9)) {
            for (int i = 0; i < rozmiar_m; i++) {
                plansza[x][y + i] = 'x';
            }
            break;
        }
        //lewo
        else if (kierunek == 1 && (x >= 0 && x <= 9) && (y-rozmiar_m > 0 && y <= 9)) {
            for (int i = 0; i < rozmiar_m; i++) {
                plansza[x][y - i] = 'x';
            }
            break;
        }
        //gora
        else if (kierunek == 2 && (x-rozmiar_m > 0  && x <= 9) && (y >= 0 && y <= 9)) {
            for (int i = 0; i < rozmiar_m; i++) {
                plansza[x - i][y] = 'x';
            }
            break;
        }
        //dol
        else if (kierunek == 3 && (x >= 0 && x+rozmiar_m < 9) && (y >= 0 && y <= 9)) {
            for (int i = 0; i < rozmiar_m; i++) {
                plansza[x + i][y] = 'x';
            }
            break;
        }
    }
}
	\end{lstlisting}
	
	\subsection*{Testy}
	Kilka przykładowych działań programu dla różnych losowych punktów:
	\begin{center}
	    \includegraphics{Screenshot_5.png}
	\end{center}
	\newline
	\begin{center}
	    \includegraphics{Screenshot_6.png}
	\end{center}
	\begin{center}
	    \includegraphics{Screenshot_7.png}
	\end{center}
	\begin{center}
	    \includegraphics{Screenshot_8.png}
	\end{center}
	\begin{center}
	    \includegraphics{Screenshot_9.png}
	\end{center}
	
	\subsection*{Wnioski}
	Program działa poprawnie natomiast na pewno można nanieść poprawki pod względem długości samego kodu, który z pewnością mógłby być krótszy. Jak i również można by ograniczyć ilość użytych w programie instrukcji warunkowych IF.
	
\end{document}
